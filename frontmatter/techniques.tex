\begin{enumerate}

    \item Read the question twice! Often we rush towards the solution and sometimes it happens that this is not what the question is asking. Therefore, a good strategy is to read the question twice. There is beautiful quote that says: \textit{Understanding a question is half an answer.}    
    
    \item Recognize the meaning and formulation of the problem and make sure of the necessary concepts needed to solve the problem before rushing into the solution. Whenever possible, draw a diagram elucidating the essence of the problem; in many cases this simplifies both the search for a solution and the solution itself.  
    
    \item Solve a problem in general form, that is express the solution in terms of letter notation. A solution in the general form is particularly valuable since it makes clear the relationship between the sought quantity and the given data. An answer in the general form allows one to make a fairly accurate judgement on the correctness of the solution itself.   

   \item After obtaining the solution in the general form, check to see if the dimensions are correct. The wrong dimensions are an obvious indication of a wrong solution. 
   
   \item If possible, investigate the behavior of the solution in some extreme special cases i.e., how the system behaves at infinities and zeroes. For example, whatever extended bodies, it must turn into the well-known law of gravitational interaction of mass points as the distance between the bodies increases. Otherwise, it can be immediately inferred that the solution is wrong.
   
   \item Approximations play an important role in the selection rounds of NePhO as well as in IPhO. If the solution seems complicated or unsolvable, try to see if any approximation makes it easier and simple.   
\end{enumerate}